\documentclass[11pt,a4paper]{article}
\usepackage{amsmath}
\usepackage{amsthm}
\usepackage{amssymb}
\usepackage[margin=2cm]{geometry}
%\usepackage{thmbox}
\usepackage{graphicx}
\usepackage[dvipsnames,usenames]{color}
\usepackage{url}
\usepackage{comment}
%\usepackage{enumerate}
%\usepackage{titlesec}
%\usepackage{Rvector}
%\usepackage{mathabx}
\newcommand{\qrq}{\quad\Rightarrow\quad}
\newcommand{\qarq}{\quad&\Rightarrow\quad}
\newcommand{\alp}{\alpha}
\newcommand{\claim}{{\underline{\it Claim:}}~~}
\newcommand{\dbR}{\mathbb{R}}
\newcommand{\ndimr}{\mathbb{R}^n}
\newcommand{\vare}{\varepsilon}
\newcommand{\since}{\because\;}
\newcommand{\hence}{\therefore\;}
\newcommand{\en}{\par\noindent}
\newcommand{\fn}{\footnotesize}

\newcommand{\sect}[2]{#1~~{\mdseries\tiny(#2)}}

\renewcommand{\(}{\left(}
\renewcommand{\)}{\right)}

\let \ds=\displaystyle

\usepackage{xeCJK}
\setCJKmainfont[AutoFakeBold=5,AutoFakeSlant=.4]{標楷體}

%\usepackage{fancyhdr}
%\pagestyle{fancy}
%\renewcommand{\headrulewidth}{0pt}

\renewcommand{\thesection}{Lecture \arabic{section}}
\renewcommand{\thesubsection}{\Roman{subsection}}

\usepackage[T1]{fontenc}

%%%% F U N C T I O N %%%%%
\newcommand{\abs}[1]{\left|#1\right|}
\newcommand{\norm}[1]{\left\|#1\right\|}
\newcommand{\inn}[1]{\left<#1\right>}
\newcommand{\f}[1]{f\!\left(#1\right)}
\newcommand{\g}[1]{g\!\left(#1\right)}
\newcommand{\h}[1]{h\!\left(#1\right)}
\newcommand{\x}[1]{x\!\left(#1\right)}
\newcommand{\D}[1]{D\!\left(#1\right)}
\newcommand{\N}[1]{N\!\left(#1\right)}
\renewcommand{\P}[1]{P\!\left(#1\right)}
\newcommand{\R}[1]{R\!\left(#1\right)}
\newcommand{\V}[1]{V\!\left(#1\right)}
\newcommand{\function}[2]{#1\!\left(#2\right)}
\newcommand{\functions}[2]{\left(#1\right)\!\left(#2\right)}

\definecolor{light-gray}{gray}{0.95}
\newcommand{\textfil}[1]{\colorbox{light-gray}{\large\color{Red} #1}}


\renewcommand{\title}{Real Analysis II:\quad Homework 01}
\renewcommand{\author}{104021615 黃翊軒}
\renewcommand{\maketitle}{\begin{center}\textbf{\Large\title}\\[6pt] {\author}\\[6pt] {\color{Gray}\footnotesize March 01, 2016}\end{center}}
\newcommand{\blue}[1]{{\color{blue}#1}}


\renewcommand{\labelenumi}{(\alph{enumi})}

\newcommand{\Exercise}[2]{\textbf{Exercise #1.} \textit{#2}}
\newtheorem{exercise}{Exercise}

%\parskip=11pt

\begin{document}

  \maketitle

  \setcounter{exercise}{1}
  
  \begin{exercise}
  	P96.
  \end{exercise}
  
  \begin{proof}
	\begin{enumerate}
		\item Let $h_1(x,y) = f(x)$. As a function on $\mathbb{R}^{2n}$, $h_1$ is measurable since $f(x):\mathbb{R}^n \rightarrow \mathbb{R} \cup \{\pm \infty \}$.
		And for any $a \in \mathbb{R}$, we have
		\begin{align*}
		\{(x,y)\in \mathbb{R}^{2n}:h_1(x,y)>a \} = \{x\in \mathbb{R}^n:f(x)>a \}\times \mathbb{R}^n
		\end{align*}
		By viewing $\mathbb{R}^{n} = \underbrace{\mathbb{R}\times \cdots \times \mathbb{R}}_n$ and using Lemma 5.2, the RHS is a measurable set in $\mathbb{R}^{2n}$.
		Similarly, the function $h_2(x,y) = g(y)$ is also a measurable function on $\mathbb{R}^{2n}$. Then by Theorem 4.10, we know that
		\begin{align*}
		h_1(x,y)\cdot h_2(x,y) = f(x)g(y) : \mathbb{R}^{2n} \to \mathbb{R}\cup\{\pm \infty\}
		\end{align*}
		is also a measurable function on $\mathbb{R}^{2n}$.
		
		\item Given $E_1, E_2$, both are measurable in $\mathbb{R}^n$. Since $\chi_{E_1}(x)\cdot \chi_{E_2}(y) = \chi_{E_1\times E_2}(x,y)$, by above we know that $\chi_{E_1\times E_2}(x,y)$ is a measurable function on $\mathbb{R}^{2n}$. Hence the set $E_1\times E_2$ is measurable in $\mathbb{R}^{2n}$. By Tonelli's theorem, 
		\begin{align*}
		|E_1\times E_2| &= \int_{E_1 \times E_2}\chi_{E_1\times E_2}(x,y)dxdy \\
		&= \int_{\mathbb{R}^{n}\times \mathbb{R}^{n}}\chi_{E_1\times E_2}(x,y)dxdy \\
		&= \int_{\mathbb{R}^n}\left[ \int_{\mathbb{R}^n}\chi_{E_1\times E_2}(x,y)dx\right]dy \\
		&= \int_{\mathbb{R}^n}\left[\int_{\mathbb{R}^n}\chi_{E_1}(x)\cdot \chi_{E_2}(y)dx\right]dy \\
		&=	\int_{\mathbb{R}^n}\chi_{E_1}(x)dx  \int_{\mathbb{R}^n}\chi_{E_2}(y)dy \\
		&= \int_{E_1}\chi_{E_1}(x)dx  \int_{E_2}\chi_{E_2}(y)dy \\
		&= |E_1||E_2|
		\end{align*}
	\end{enumerate}
  \end{proof}


\end{document} 