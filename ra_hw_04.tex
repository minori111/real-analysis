\documentclass[11pt,a4paper]{article}
\usepackage{amsmath}
\usepackage{amsthm}
\usepackage{amssymb}
\usepackage{mathabx}
\usepackage{mathrsfs}
\usepackage[margin=2cm]{geometry}
%\usepackage{thmbox}
\usepackage{graphicx}
\usepackage[dvipsnames,usenames]{color}
\usepackage{url}
\usepackage{comment}
%\usepackage{esint} %重積分
%\usepackage{enumerate}
%\usepackage{titlesec}
%\usepackage{Rvector}
%\usepackage{mathabx}
\newcommand{\qrq}{\quad\Rightarrow\quad}
\newcommand{\qarq}{\quad&\Rightarrow\quad}
\newcommand{\alp}{\alpha}
\newcommand{\claim}{{\underline{\it Claim:}}~~}
\newcommand{\dbR}{\mathbb{R}}
\newcommand{\ndimr}{\mathbb{R}^n}
\newcommand{\vare}{\varepsilon}
\newcommand{\since}{\because\;}
\newcommand{\hence}{\therefore\;}
\newcommand{\en}{\par\noindent}
\newcommand{\fn}{\footnotesize}

\newcommand{\sect}[2]{#1~~{\mdseries\tiny(#2)}}

\renewcommand{\(}{\left(}
\renewcommand{\)}{\right)}
\renewcommand{\[}{\left[}
\renewcommand{\]}{\right]}

\let \ds=\displaystyle

\usepackage{xeCJK}
\setCJKmainfont[AutoFakeBold=5,AutoFakeSlant=.4]{標楷體}

%\usepackage{fancyhdr}
%\pagestyle{fancy}
%\renewcommand{\headrulewidth}{0pt}

\renewcommand{\thesection}{Lecture \arabic{section}}
\renewcommand{\thesubsection}{\Roman{subsection}}

\usepackage[T1]{fontenc}

\newcommand{\Rn}{\mathbb{R}^n}
\newcommand{\intR}{\int_{\mathbb{R}}}
\newcommand{\intRn}{\int_{\mathbb{R}^n}}
\newcommand{\SRn}{\mathscr{S}(\mathbb{R}^n)}
\newcommand{\EXP}{e^{-{1\over1-\left|x\right|^2}}}
\newcommand{\OMA}{\left(1-\left|x\right|^2\right)}

%%%% F U N C T I O N %%%%%
\newcommand{\abs}[1]{\left|#1\right|}
\newcommand{\norm}[1]{\left\|#1\right\|}
\newcommand{\inn}[1]{\left<#1\right>}
\newcommand{\f}[1]{f\!\left(#1\right)}
\newcommand{\g}[1]{g\!\left(#1\right)}
\newcommand{\h}[1]{h\!\left(#1\right)}
\newcommand{\x}[1]{x\!\left(#1\right)}
\newcommand{\D}[1]{D\!\left(#1\right)}
\newcommand{\N}[1]{N\!\left(#1\right)}
\renewcommand{\P}[1]{P\!\left(#1\right)}
\newcommand{\R}[1]{R\!\left(#1\right)}
\newcommand{\V}[1]{V\!\left(#1\right)}
\newcommand{\function}[2]{#1\!\left(#2\right)}
\newcommand{\functions}[2]{\left(#1\right)\!\left(#2\right)}

\definecolor{light-gray}{gray}{0.95}
\newcommand{\textfil}[1]{\colorbox{light-gray}{\large\color{Red} #1}}


\renewcommand{\title}{Real Analysis II:\quad Homework 04}
\renewcommand{\author}{104021615 黃翊軒}
\renewcommand{\maketitle}{\begin{center}\textbf{\Large\title}\\[6pt] {\author}\\[6pt] {\color{Gray}\footnotesize March 28, 2016}\end{center}}
\newcommand{\blue}[1]{{\color{blue}#1}}

\newcommand{\bfL}{\text{\boldmath $L$}}
\newcommand{\bfone}{\text{\boldmath $1$}}
\renewcommand{\D}[3]{\frac{\partial^{#2}{#3}}{\partial{#1}^{#2}}}
\renewcommand{\d}[3]{\frac{d^{#2}{#3}}{d{#1}^{#2}}}
\newcommand{\japan}{\left<x\right>}


\renewcommand{\labelenumi}{(\alph{enumi})}

\newcommand{\Exercise}[2]{\textbf{Exercise #1.} \textit{#2}}
\newtheorem{exercise}{Exercise}

%\parskip=11pt

\DeclareMathOperator*{\esssup}{ess\,sup}
\DeclareMathOperator*{\essinf}{ess\,inf}

\begin{document}

  \maketitle

  %\noindent\textbf{\large CHAPTER 2. ~Tempered distributions}

  \setcounter{exercise}{4}
  
  \begin{exercise}
  	p.143
  \end{exercise}
  \begin{proof}
  	We may assume $N_{p_1}[f]<+\infty$, otherwise the result is trivial by Theorem 8.2.
  	
  	Let $q = p_2/p_1 > 1$, then the conjugate exponent of $q$ is $p_2/(p_2-p_1)$. By H\"older's Inequality, we get
  	\begin{align*}
  	\|f\|_{p_1}^{p_1}&=\int_E|f|^{p_1} = \int_E|f|^{p_1}\times 1\\
  	&\le \(\int_E|f|^{p_1q}\)^{1/q}\(\int_E 1^{p_2/(p_2-p_1)}\)^{(p_2-p_1)/p_2}\\
  	&= \(\int_E|f|^{p_2}\)^{p_1/p_2}|E|^{(p_2-p_1)/p_2}
    \end{align*}
    Taking both sides to the power $1/p_1$, then
    \begin{align*}
    \|f\|_{p_1}&\le\(\int_E|f|^{p_2}\)^{1/p_2}|E|^{(p_2-p_1)/p_1p_2}\\
    &=\|f\|_{p_2}|E|^{(p_2-p_1)/p_1p_2}
    \end{align*}
    Multipling both sides $1/|E|^{1/p_1}$, we have
    $$
    N_{p_1}[f]=\frac{1}{|E|^{1/p_1}}\|f\|_{p_1}\le\frac{1}{|E|^{1/p_2}}\|f\|_{p_2}=N_{p_2}[f]
    $$
    \\
    
    Next we check other three properties.

    From  Minkowski's Inequilty,
    \begin{align*}
    	N_p[f+g] &= \frac{1}{|E|^{1/p}}\|f+g\|_p\\
    	&\le \frac{1}{|E|^{1/p}}\(\|f|_p+\|g\|_p\)\\
    	&= \frac{1}{|E|^{1/p}}\|f\|_p+\frac{1}{|E|^{1/p}}\|g\|_p\\
    	&= N_p[f]+N_p[g]
    \end{align*}
    
    From H\"older's Inequality,
    \begin{align*}
    N_1[fg] = \frac{1}{|E|}\int|fg| &\le \frac{1}{|E|^{1/p+1/p'}}\int|f|\int|g|\\
    &= N_p[f]N_{p'}[g]
    \end{align*}
    
    From Theorem 8.1,
    \begin{align*}
    \lim_{p\rightarrow\infty}N_p[f] &= \lim_{p\rightarrow\infty}\frac{1}{|E|^{1/p}}\|f\|_p\\
    &=\lim_{p\rightarrow\infty}\frac{1}{|E|^{1/p}}\cdot\lim_{p\rightarrow\infty}\|f\|_p\\
    &=1\cdot\lim_{p\rightarrow\infty}\|f\|_p\\
    &=\|f\|_\infty
    \end{align*}
  \end{proof}
  
  \setcounter{exercise}{6}
  
  \begin{exercise}
  	p.143
  \end{exercise}
  \begin{proof}
  	We quickly have 
  	$$
  	\|f\|_p=\(\int_{(0,\varepsilon^p)}|1|^p\)^{1/p} = \varepsilon
  	$$
  	and similarly $\|g\|_p=\varepsilon$. However,
  	$$
  	\|f/2+g/2\|_p = \(\int_{(0,2\varepsilon^p)}|1/2|^p\)^{1/p} = \(2\vare^p2^{-p}\)^{1/p}=2^{\frac{1}{p}-1}\varepsilon>\varepsilon
  	$$
  	since $1/p>1$. So the neighborhood $B_{\vare+\eta}(0)$ is not convex for sufficiently small $\eta$.
  \end{proof}
  
  \setcounter{exercise}{8}
  
  \begin{exercise}
  	p.143
  \end{exercise}
  \begin{proof}
  	Suppose $\essinf_Ef = 0$. Then for every $\alpha>0$ we have $|\{x\in E:f(x)<\alp\}|>0$. Thus, for every $0<\beta<+\infty$ we have $|\{x\in E:1/f(x)>\beta\}|>0$, so $\esssup_E 1/f = +\infty$. We may interpret $+\infty^{-1} = 0$, so the proposition still holds.\\
  	
  	Now suppose $\essinf_Ef>0$, so there exists $\alp>0$ such that $|\{x\in E:f(x)<\alp\}|>0$. Then
  	\begin{align*}
  	\essinf_E f &= \sup\{\alp>0:|\{x\in E:f(x)<\alp\}|>0\}\\
  	&=\sup\{\alp>0:|\{x\in E:1/f(x)>1/\alp\}|>0\}\\
  	&=\sup\{1/\beta>0:|\{x\in E:1/f(x)>\beta\}|>0\}\\
  	&=\(\inf\{\beta>0:|\{x\in E:1/f(x)>\beta\}|>0\}\)^{-1}\\	
  	&=\(\esssup_E (1/f)\)^{-1}
  	\end{align*}
  \end{proof}

  \begin{exercise}
  	p.143
  \end{exercise}
  \begin{proof}
  	Let $\vare_k = 1/k$, then by Lemma 3.22. We have for every $\vare_k$, there exists a closed set $F_k$ such that $$\vare_{k+1}<|E- F_k|<\vare_k.$$ Note that such $F_k$ is possible by reversely using Lemma 3.22 again if needed. Then we have a sequence of sets of strictly decreasing measure $|E-F_{k+1}|<|E-F_k|$. 
  	
  	The difference of any two of these sets must then have positive measure. Let $A$ be the set of all possible differences of sets $|E-F_k|$. The set of all possible unions of sets in $A$ is the power set of $A$, $P(A)$. Note that the power set is uncountable. Taking the characteristic function of any two distinct sets in $P(A)$ gives $$\|\chi_{B_1}-\chi_{B_2}\|_\infty = 1,$$ because $B_1$ and $B_2$ must differ by a set of positive measure. 
  	
  	Since there are an uncountable number of such functions and their norms are all different by 1, they can be contained in disjoint open balls of radius $1/3$ in the space $L^\infty(E)$. Then the space $L^\infty(E)$ cannot be separable, because no countable set can intersect this uncountable family of disjoint balls.
  \end{proof}

\end{document} 